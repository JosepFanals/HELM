ARRILLAGA, J., WATSON, N.R. Power System Harmonics. University of Canterbury. John Wiley \& Sons Ltd. 2003. % vist des de google books

BARRERO, F. Sistemas de energía eléctrica. Ediciones Paraninfo. Madrid. 2004.

BONINI, N.A., MATARUCCO, R.R., ALVES, D.A. Technique for Continuation Power Flow Using the Flat Start and for Ill-Conditioned Systems. World Journal Control Science and Engineering. Vol. 3. No. 1. p. 1-7. 2015.

BREZINSKI, C. A derivation of extrapolation algorithms based on error estimates. Journal of Computational and Applied Mathematics. Vol. 66. No. 1-2. p. 5-26. 1996.

CHRISTIE, R. Power systems test case archive. University of Washington. Department of Electrical Engineering. (\url{http://labs.ece.uw.edu/pstca}, 30 d'abril de 2020)

DRONAMRAJU, A., LI, S., LI, Q., LI, Y., TYLAVSKY, D., SHI, D., WANG, Z. Implications of Stahl's Theorems to Holomorphic Embedding Pt. 2: Numerical Convergence. 2020.

ELEQUANT. Company. Corporate Overview. (\url{http://elequant.com/company}, 22 de febrer de 2020)

FLISCOUNAKIS, S., PANCIATICI, P., CAPITANESCU, F., WEHENKEL, L. Contingency ranking with respect to overloads in very large power systems taking into account uncertainty, preventive and corrective actions. IEEE Transactions on Power Systems. Vol 28. No. 4. p. 4909-4917. 2013. % citar-ho en els resultats

GLOVER, J.D., SARMA, M.S., OVERBYE, T.J. Power System Analysis and Design. PWS Publishing Company. Boston. 2008. % apareix a la biblioteca

GRIFFITHS, D.F. Numerical Analysis. Proceedings of the Biennial Conference Held at Dundee. Springer. 1983.

HARIHARAN, G., VARWANDKAR, S.D., GUPTA, P.P. Modular Load Flow for Restructured Power Systems. Lecture Notes in Electrical Engineering. Springer. 2016. % vist des de google books

IWAMOTO, S., TAMURA, Y. A Load Flow Calculation Method for Ill-Conditioned Power Systems. IEEE Transactions on Power Apparatus and Systems. Vol. PAS-100. No. 4. p. 1736-1743. 1981.

JOSZ, C., FLISCOUNAKIS, S., MAEGTH, J., PANCIATICI, P. AC Power Flow Data in MATPOWER and QCQP Format: iTesla, RTE Snapshots, and PEGASE. 2016. % citar-ho en els resultats

KOTHARI, D.P., NAGRATH, I.J. Modern Power Systems Analysis. Tata McGraw Hill. Nova Dehli. 2011.

LAGACE, P.J., VUONG, M.H., KAMWA, I. Improving power flow convergence by Newton Raphson with a Levenberg-Marquardt method. IEEE Power and Energy Society General Meeting. Conversion and Delivery of Electrical Energy in the 21st Century. Pittsburgh. p. 1-6. 2008.

MERCER, G.N., ROBERTS, A. A centre manifold description of contaminant dispersion in channels with varying row properties. SIAM Journal on Applied Mathematics. Vol. 50. No. 6. p. 1547-1565. 1990. % citat per la wikipedia

MESAS, J.J. Estudio y caracterización de cargas no lineales. Tesis doctoral. Universitat Politècnica de Catalunya. Novembre 2009.

NUMPY. User Guide. NumPy v1.18 Manual. (\url{https://numpy.org/doc/1.18/user/index.html}, 15 de gener de 2020)

PANDAS. User Guide. Pandas 1.0.0 Documentation. (\url{https://pandas.pydata.org/pandas-docs/version/1.0.0/user_guide/index.html#user-guide}, 23 de febrer de 2020)

PEÑATE, S. GridCal Documentation. (\url{https://gridcal.readthedocs.io/}, 17 de març de 2020a)

PEÑATE, S. GridCal. Grids and Profiles. (\url{https://github.com/SanPen/GridCal/tree/master/}, 4 d'abril de 2020b)

PRESS, W.H., TEUKOLSKY, S.A., VETTERLING, W.T., FLANNERY, B. P. Numerical Recipes. The Art of Scientific Computing. Cambridge. 2007. % el primer resultat de buscar pel google

% RAO, S., FENG, Y., TYLAVSKY, D., SUBRAMANIAN, M.K. The Holomorphic Embedding Method Applied to the Power-Flow Problem. IEEE Transactions on Power Systems. Vol. 31. No. 5. p. 3816-3828. 2016.

RAO, S. Exploration of a Scalable Holomorphic Embedding Method Formulation for Power System Analysis Applications. Doctoral thesis. Arizona State University. Agost 2017.

RASHID, A. Harmonic load flow formulation and numerical resolution. Doctoral thesis. Universitat Politècnica de Catalunya. Novembre 2018.

SCHMIDT, B. Implementation and Evaluation of the Holomorphic Embedding Load Flow Method. Master thesis. Institute of Power Transmission Systems. Technical University of Munich. Març 2015.

SCIPY. Linear Algebra (scipy.linalg). SciPy v1.4.1 Reference Guide. (\url{https://docs.scipy.org/doc/scipy/reference/tutorial/linalg.html}, 22 de febrer de 2020)

SHANKS, D. Nonlinear transformations of divergent and slowly convergent sequences. Journal of Mathematical Physics. Vol. 34. No. 1-4. p. 1-42. 1955. % he seguit la wikipedia, i a la wikipedia se cita

STAHL, H. On the convergence of generalized Padé approximants. Constructive Approximation. Vol. 5. p. 221-240. 1989.

STAHL, H. The Convergence of Padé Approximants to Functions with Branch Points. Journal of Approximation Theory. Vol. 91. No. 2. p. 139-204. 1997.

STOTT, B., ALSAC, O. Fast decoupled load flow. IEEE Transactions on Power Apparatus and Systems. Vol. PAS-93. No. 3. p. 859-869. 1974.

SUBRAMANIAN, M.K. Application of Holomorphic Embedding to Power-Flow Problem. Master thesis. Arizona State University. Agost 2014.

TRIAS, A. The Holomorphic Embedding Load Flow method. IEEE Power and Energy Society General Meeting. San Diego. p. 1-8. 2012.

TRIAS, A. Sigma algebraic approximants as a diagnostic tool in power networks. United States Patent Application Publication. 2014.

TRIAS, A. The Holomorphic Embedding Load-Flow Method. Foundations and Implementations. Foundations and Trends in Electric Energy Systems. Vol. 3. No. 3-4. p. 140-370. 2018.

TRIPATHY, S.C., DURGA, P.G., MALIK, O.P., HOPE, G.S. Load-Flow for Ill-Conditioned Power Systems by a Newton-Like Method. IEEE Transactions on Power Apparatus and Systems. Vol. PAS-101. No. 10. p. 3648-3657. 1982.

VON ROSSUM, G. PEP 8. Style Guide for Python Code. (\url{https://www.python.org/dev/peps/pep-0008/}, 29 de març de 2020)

WENIGER, E.J. Nonlinear sequence transformations for the acceleration of convergence and the summation of divergent series. Computer Physics Reports. Vol. 10. No. 5-6. p. 189-371. 1989.

ZIMMERMAN, R.D., MURILLO-SÁNCHEZ, C.E., THOMAS, R.J. MATPOWER: Steady-State Operations, Planning and Analysis Tools for Power Systems Research and Education. IEEE Transactions on Power Systems. Vol. 26. No. 1. p. 12–19. 2011.


